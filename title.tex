\documentclass[13pt]{article}
\usepackage{ctex}
\usepackage{verbatim}
\usepackage{geometry}
\usepackage{van-de-la-sehen}
%\usepackage{section}
\title{Lorem ipsum dolor sit amet.}
\author{perry}
\date{\today}
\geometry{left=3cm,right=3cm,top=3cm,bottom=3cm}
\begin{document}
	\maketitle
	\indent
	\parskip 0pt 

	\pagestyle{myheadings}
	\markright{jsdijsdi}
	\begin{abstract}
		\lipsum[1]
	\end{abstract}
\begin{abstract}
	内容...
\end{abstract}
{\bf keywords: ex,imperdiet\footnote{text},ultricies,non,eu,est}\\
\textheight=100\baselineskip

\begin{theorem}[Parseval Indentity]%
	$f\in L^2$,
	\begin{equation*}
	\frac{a_0^2}{2} + \sum a_i^2 = \frac{1}{\pi}\int_{-\pi}^\pi f^2\pare{x}\,\rd{x}.
	\end{equation*}
\end{theorem}
\begin{pitfall}

	kjdshgkjdshgkjhfdskjg
\end{pitfall}

\begin{finale}
	hfkjdsahgdsgkjhfdskjghdfs
\end{finale}

\begin{reflex}{This is a reflex}{Das_ist_eine_reflex}
	hdsakghkjdshgkjdshgkjdhfsgkj
\end{reflex}

\begin{equation*}
	E=mc^2,\quad\text{其中} c\text{\textit{是光速}}.
\end{equation*}

\begin{equation*}
    \int \rec{\sqrt{1-x^2}}\,\rd x.
\end{equation*}

$\cE=-\rd{\Phi}/\rd{t}$. 
$\cA$. 
$\R$. 
$\N$.
$\mathbb{R}$. 
$\mathrm{d}x$.
$\bC$.
$\vvu$, $\vr$

\part{Lorem ipsum}
\section{Lorem ipsum}
\subsection{Lorem ipsum}
\subsubsection{Lorem ipsum}
\normalsize Lorem ipsum dolor sit amet\thanks{"Neque porro quisquam est qui dolorem ipsum quia dolor sit amet, consectetur, adipisci velit..."}, consectetur adipiscing elit. Nulla varius diam ut nisi lacinia, a congue mauris ultricies. Nunc eu nisi ipsum. Morbi ex massa, pharetra eget porttitor vel, pharetra ut \footnote{text }mi. Integer a odio accumsan, dapibus tortor nec, pharetra dui. Suspendisse potenti. In et vestibulum leo. Etiam vehicula est.
Pellentesque
\subsubsection[short title]{title}
 lacinia augue ut dui porttitor, sit amet consequat ex mattis. Mauris lectus nulla, condimentum nec dui rutrum, placerat pulvinar nunc. Aenean gravida elementum tempus.
 \subsection{titile} Etiam eu odio quis urna commodo porttitor. Donec egestas fermentum augue, sit amet dapibus mi dignissim ut. Donec facilisis vitae magna eu tincidunt. Morbi tincidunt euismod diam quis varius. Fusce sapien dolor, molestie ac feugiat ac, suscipit in tellus. Etiam lectus ante, elementum id magna ut, feugiat 
 \subsubsection{title}
 vulputate augue. Aenean a nunc et erat vestibulum sollicitudin. Aliquam maximus tempus metus eget porta.
\twocolumn[null]
\paragraph{title}
\normalsize Phasellus pulvinar finibus arcu nec pulvinar. Quisque ac ultrices sapien. Proin turpis libero, finibus molestie consectetur eget, tincidunt vitae risus. Vivamus ultricies\footnote{text} sagittis ultrices. Sed at odio tincidunt, vestibulum est et, commodo velit. Maecenas fringilla aliquet fermentum. Integer sit amet porttitor nibh, a eleifend augue. Nunc ut orci felis. Proin accumsan ipsum ut egestas fringilla. Nulla \section{title}
orci eros, pellentesque ac euismod ac, volutpat aliquet est. Nunc id nulla lectus. Donec vitae faucibus risus, ut dignissim nibh. Pellentesque venenatis nec diam eget vehicula. Proin pulvinar maximus magna, et varius turpis suscipit at. Aliquam mollis malesuada nunc, a ultrices sem 
\section{title}
sollicitudin non.
\[\forall x,y,z\in {\cal N}, \not \exists x,y,z ,s.t. ~x^n+y^n=z^n, 3\leq n\]
 Praesent eleifend metus non elit sodales, ut vulputate libero placerat. Proin imperdiet lacus elit, semper elementum dui porta placerat. Phasellu
 \onecolumn
\setcounter{equation}{1000}
\begin{equation}
	\iint_{S}(\bigtriangledown\times{\vec{v})\cdot d\vec{S} =\oint_{\partial S}\vec{v}\cdot d\vec{S}}
\end{equation}
\begin{equation}
\biguplus\bigcap
\bigotimes\prod_{m,n,n}^{\sum_{n=10}^{{20}=\oint=\oiint^{\iiint\iiiint}}}
\end{equation}
\[\int x^{\sin (x)} \, dx\]
\[\left\{ \begin{array}{cc}
		\iint_{S}(\bigtriangledown\times{\vec{v})\cdot d\vec{S} =\oint_{\partial S}\vec{v}\cdot d\vec{S}} & 2332 \\
		\iint_{S}(\bigtriangledown\times)=	\iint_{S}(\bigtriangledown & \times{\vec{v})\cdot d\vec{S} =\oint_{\partial S}\vec{v}\cdot d\vec{S}} 
		
\end{array}\right.\]



\begin{table}
	\begin{picture}(120,120)
	\put(1,10){\line(1,0){120}}
	\multiput(0,100)(2,-1){40}{\circle{10}}
	\put(50,50){\circle{30}}
	\put(50,50){\circle*{30}}
	\put(50,50){\vector(3,4){40}}
	\put(50,50){\vector(-3,-4){40}}
	\put(20,60){\oval(33,98)}
	\multiput(0,0)(1,1){70}{\line(5,2){5}}
	\multiput(100,0)(-4,3){30}{\circle*{1}}
	\end{picture}
	\caption{a sample of picture}
\end{table}

\end{document}
